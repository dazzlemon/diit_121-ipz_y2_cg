\documentclass{article}
\usepackage{amsmath}
\usepackage{amsfonts}
\usepackage{amssymb}
\usepackage{gensymb}

\begin{document}

\section{Simple affine transformations in 3D}

\subsection{Translate}
by $\Delta$ (x, y, z)
\begin{gather}
	\begin{bmatrix}
		x'\\
		y'\\
		z'\\
		\_
	\end{bmatrix}
	=
	\begin{bmatrix}
		1 & 0 & 0 & \Delta x\\
		0 & 1 & 0 & \Delta y\\
		0 & 0 & 1 & \Delta z\\
		0 & 0 & 0 & 1
	\end{bmatrix}
	\begin{bmatrix}
		x\\
		y\\
		z\\
		1
	\end{bmatrix}
\end{gather}

\subsection{Scale}
about origin
by S (x, y, z)
\begin{gather}
	\begin{bmatrix}
		x'\\
		y'\\
		z'\\
		\_
	\end{bmatrix}
	=
	\begin{bmatrix}
		S_x & 0   & 0   & 0\\
		0   & S_y & 0   & 0\\
		0   & 0   & S_z & 0\\
		0   & 0   & 0   & 1
	\end{bmatrix}
	\begin{bmatrix}
		x\\
		y\\
		z\\
		1
	\end{bmatrix}
\end{gather}

\subsection{Rotate}\label{rotations}
about $O_z$
by $\theta$
\begin{gather}
		\begin{bmatrix}
		x'\\
		y'\\
		z'\\
		\_
	\end{bmatrix}
	=
	\begin{bmatrix}
		\cos \theta & -\sin \theta & 0 & 0\\
		\sin \theta &  \cos \theta & 0 & 0\\
		0           & 0            & 1 & 0\\
		0           & 0            & 0 & 1
	\end{bmatrix}
	\begin{bmatrix}
		x\\
		y\\
		z\\
		1
	\end{bmatrix}
\end{gather}
\\
about $O_x$
by $\theta$
\begin{gather}
		\begin{bmatrix}
		x'\\
		y'\\
		z'\\
		\_
	\end{bmatrix}
	=
	\begin{bmatrix}
		1 & 0           & 0            & 0\\
		0 & \cos \theta & -\sin \theta & 0\\
		0 & \sin \theta &  \cos \theta & 0\\
		0 & 0           &  0           & 1
	\end{bmatrix}
	\begin{bmatrix}
		x\\
		y\\
		z\\
		1
	\end{bmatrix}
\end{gather}
\\
about $O_y$
by $\theta$
\begin{gather}
		\begin{bmatrix}
		x'\\
		y'\\
		z'\\
		\_
	\end{bmatrix}
	=
	\begin{bmatrix}
		\cos \theta & 0 & -\sin \theta & 0\\
		0           & 1 & 0            & 0\\
		\sin \theta & 0 &  \cos \theta & 0\\
		0           & 0 & 0            & 1
	\end{bmatrix}
	\begin{bmatrix}
		x\\
		y\\
		z\\
		1
	\end{bmatrix}
\end{gather}

\clearpage
\section{Axonometrix projections}
$T_{axonometric_{plane}} = T_{rot_1} * T_{rot_2} * T_{orto_{plane}}$
\\e.g.
\begin{gather}T_{axonometric_{z=0}} = T_{rot_y} * T_{rot_x} * T_{orto_{z=0}}\end{gather}
\begin{gather}T_{axonometric_{y=0}} = T_{rot_x} * T_{rot_z} * T_{orto_{y=0}}\end{gather}
\begin{gather}T_{axonometric_{x=0}} = T_{rot_z} * T_{rot_y} * T_{orto_{x=0}}\end{gather}
\\where $T_{rot_{axis}}$ is one of matrices from Section~\ref{rotations}
\\and $T_{orto_{plane}}$ is ortographic projection onto a plane
\\e.g. $T_{orto_{z=0}} = 
\begin{bmatrix}
	1 & 0 & 0 & 0\\
	0 & 1 & 0 & 0\\
	0 & 0 & 0 & 0\\
	0 & 0 & 0 & 1
\end{bmatrix}
$
\\so we get $T_{axonometric_{z=0}} = 
\begin{bmatrix}
	\cos \theta &  \sin \theta * \cos \phi & 0 & 0\\
	0           &  \cos \phi               & 0 & 0\\
	\sin \theta & -\cos \theta * \sin \phi & 0 & 0\\
	0           & 0                        & 0 & 1
\end{bmatrix}
$
\\where $\theta$ and $\phi$ are rotation angles around $O_x$ and $O_y$ accordingly
\\trimetric, isometric and dimetric have different ratios of distortion coefficients ($K_x, K_y, K_z$)
\\where
\\
$
	K_{xx=0} = \sqrt{{\sin}^2 \psi - {\cos}^2 \psi * {\sin}^2 \theta}\\
	K_{yx=0} = \sqrt{{\cos}^2 \psi + {\sin}^2 \psi * {\sin}^2 \theta}\\
	K_{zx=0} = \sqrt{{\cos}^2 \theta}\\
	\\
	K_{xy=0} = \sqrt{{\cos}^2 \psi}\\
	K_{yy=0} = \sqrt{{\sin}^2 \phi - {\cos}^2 \phi * {\sin}^2 \psi}\\
	K_{zy=0} = \sqrt{{\cos}^2 \phi + {\sin}^2 \phi * {\sin}^2 \psi}\\
	\\
	K_{xz=0} = \sqrt{{\cos}^2 \theta + {\sin}^2 \theta * {\sin}^2 \phi}\\
	K_{yz=0} = \sqrt{{\cos}^2 \phi}\\
	K_{zz=0} = \sqrt{{\sin}^2 \theta - {\cos}^2 \theta * {\sin}^2 \phi}
$

\clearpage
\subsection{Isometric}
Isometric projections are commonly used in technical drawings and used to be used in some computer game graphics.
In an isometric projection the three axes appear $120\degree$ drawings and used to from each other and are equally foreshortened.
It can be achieved by rotating an object $45 \degree$ in the plane of the screen and $\sim 35.3 \degree (\arctan (1 / \sqrt 2))$ through the horizontal axis.
\\$K_x = K_y = K_z$
\subsection{Dimetric}
$K_x = K_y \neq~K_z$
\\$K_y = K_z \neq~K_x$
\\$K_z = K_x \neq~K_y$
\end{document}
